\documentclass[a4paper,10pt]{article}
\usepackage[utf8x]{inputenc}

%opening
\title{PAPAGENO\\ Parallel Parser Generator for Operator grammars}
\author{Ermes Viviani, Alessandro Barenghi}

\begin{document}

\maketitle

\section{Introduction}
PAPAGENO is a parallel parser generator for use with Floyd operator precedence
grammars. It accepts a Bison-style description of the grammar and produces in 
output the C implementation of a parallel parser using \texttt{pthreads}.

\subsection{System requirements}

PAPAGENO currently runs on any Linux system equipped with Python 2.x.
The generated parser implementation can be compiled with any ANSI C compiler.
In order to build and execute the parser the following tools are needed:
\begin{itemize}
 \item GCC 
 \item Make
 \item GNU Flex
\end{itemize}

The generated source code employs only standard C library functions and an implementation of POSIX threads.
The code has been benchmarked on the Linux NPTL POSIX thread implementation.

The \texttt{test} directory in the codebase provides an automated regression test employing a Bison generated parser on the JSON language grammar.
to run the regression test, simply invoke make from both the main directory and the \texttt{test} directory, then subsequently run the \texttt{reg\_test.py} script.


\section{Programming Interface}

\subsection{Input grammar syntax}

The input grammar specification follows a syntax similar to the one used in
other parser generators such as Yacc/Bison and is expressed as:

\begin{verbatim}
  - grammar    = C_preamble, "%%", axiom, "%%", rules;
  - C_preamble = C_code;
  - axiom      = "%axiom", token;
  - rules      = rule, {rule};
  - rule       = token, ":", rhs, {"|", rhs}, ";";
  - rhs        = token, {token}, ["{", C_mapped_code, "}"];
  - token      = alphanum, {alphanum};
  - alphanum   = ? Every alphanumeric character ?;
\end{verbatim}


where C\_code indicates a  generic C code which will be prefixed added to the implementation 
and C\_mapped\_code express generic C code performing the semantic action associated to the rule.
In this code it is possible to refer to the semantic value of the tokens in the rule through the syntax \texttt{\$x}, 
where x takes a numeric value (counting from zero), to indicate the elements of the right hand side of the rule, 
while \texttt{\$\$} indicates the left hand side.
It is possible to include C style comments in the description of the grammar.

We follow the convention that the names of the terminals and nonterminals of the grammar cannot have '$\_$' as initial character. 

A sample grammar specification for arithmetic expressions is bundled with the generator.

\subsection{Lexicon Specification}

PAPAGENO employs Flex to generate the lexical analyzer to scan over the text.
It is necessary to write a in.l file describing the scanner in which each token is associated with
a correctly formed lex\_token. 
In order to do so the following code needs to be included in the specification file

\begin{verbatim}
%{
 #include "../src/grammar_tokens.h"
 struct lex_token {
     gr_token token;
       void* semantic_value;
 };
 extern struct lex_token* flex_token;
 %} 
\end{verbatim}

The scanner should assign the token identifier to the token field of the structure and make the semantic\_value field point to the correct semantic value of the token as follows:

\begin{verbatim}
lex_token->token = LBRACE;
flex_token->semantic_value = <pointer_to_semantic_value>; 
\end{verbatim}
and must return a value of 1

\subsection{How to call the parser}

The generated parser is invoked with the function in the source code through the function:

\begin{verbatim}
token_node *parse(int32_t threads, char *file_name);
\end{verbatim}

where \texttt{threads} is the number of threads to be used to parse the input and \texttt{file\_name} is the full pathname of the file to be parsed;
The function returns a pointer to the root of the abstract syntax tree resulting from the parsing of the input string. \textbf{}

\section{Using the generator}

The generator is invoked with the following syntax:\\
\begin{verbatim}
papageno.py [-h] [--verbose verb] --inputfile inputfile
                   [--out_header header_outpath] [--out_core source_outpath]
 \end{verbatim}

where the parameters represent:
%\begin{itemize}
% \item  \texttt{grammar\_specification} indicates the grammar specification file
% \item  \texttt{output\_dir} indicates the directory where the generated parser files are output
% \item  \texttt{-dlow}: shows low level debug information, such as C preamble, bitpacked matrix vectorized reduction tree and token mapping;
% \item  \texttt{-dhigh}: shows high level debug information, such as identified rules, precedence matrix and rewrite rules;
% \item  \texttt{-d}: is equivalent to -dlow -dhigh;
%\end{itemize}
\begin{itemize}
 \item  \texttt{-h, --help}: show an help message and exit;
 \item  \texttt{--verbose verb, -v verb}:  Verbosity level [0-2];
 \item  \texttt{--inputfile inputfile, -i inputfile}: Grammar specification file, mandatory;
 \item  \texttt{--out\_header header\_outpath}: Location where the output headerfiles should be generated. Defaults to ./include/ in the calling path;
 \item  \texttt{--out\_core source\_outpath}: Location where the output C files should be generated. Defaults to ./lib/ in the calling path.
\end{itemize}

The generator executes checks on the input grammar to assess whether it is in
the required operator form or not. In particular, it detects repeated rhs,
rules with consecutive nonterminal symbols and precedence conflicts among
terminal symbols. Whenever such a problem occurs, the generator outputs to
stdout diagnostic code that may help in transforming the grammar into the proper
form.

\section{Output files}

PAPAGENO generates in output a C implementation of the parallel operator precedence parser constituted by the following files:
\begin{itemize}
 \item \texttt{grammar\_tokens.h} contains the definitions of terminals and nonterminals of the grammar
 \item \texttt{grammar\_semantics.$\{$h,c$\}$} contain information regarding the rules of the grammar and the associated semantic actions
 \item \texttt{grammar.$\{$h,c$\}$} contain the actual structure of the grammar
 \item \texttt{matrix.h} contains the terminals/nonterminals precedence matrix employed by the parser
 \item \texttt{reduction\_tree.c} contains the reductions to be performed during the parsing process, stored as a prefix tree
 \item \texttt{rewrite\_rules.c} contains the generated rewrite sets
\end{itemize}

\section{Notes}
For the sake of portability, it is possible to define which bit represents the nonterminal flag on the integer-represented symbols.
The information is contained in code\_emission.py and bitpack.py.\\

The trie employed to store the reduction rules is vectorized in preorder as ruleId-nextSymbol-offsetOfNextSymbol
\end{document}
